\documentclass{article}
\usepackage[utf8]{inputenc}
\usepackage{verbatim}

\title{Accelerating End-to-End Data Science Workflows}
\author{Notes inspired by NVIDIA DLI}
\date{\today}

\begin{document}

\maketitle

\section{Introduction to cuDF}
\subsection{Reading and Writing Data}
The RAPIDS API provides a GPU accelerated dataframe with the \textbf{cudf} module, as shown below:

\begin{verbatim}

gdf = cudf.read_csv('/directory/path')

gdf.shape #displays shape

gdf.dtypes #display data types

\end{verbatim}

For coloquial purposes, \verb_gdf_ will be used to represent a GPU dataframe and \verb_df_ for a CPU dataframe.

\subsection{Writing to File}

The syntax for writing to csv is similar to the Pandas was of doing things:

\begin{verbatim}

datatocsv.to_csv('csvfilename.csv')


\end{verbatim}
\section{Basic Operations with cuDF}
\subsubsection{Converting data types}
Let's say I want convert an int64 to float32, I would use the \verb_.astype('float32')_ attribute to convert this dtype
For example:


\begin{verbatim}

gdf['age'] = gdf['age'].astype('float32')

\end{verbatim}



\subsubsection{Column-wise aggregations}

Column-wise aggregations take advantage of the GPU's architecture and RAPIDS' memory format. \\

For comparison, cuDF's \verb_.mean()_ attribute completes in 4ms as opposed to Pandas' 236ms!

\subsubsection{String operations}
Although strings are not a datatype traditionally associated with GPUs, cuDF supports powerful accelerated string operations.

For instance, using the titlecase function \verb_.str.title()_ completes in 24ms with cuDF versus Pandas' ~24s! \\

Some example syntax:

\begin{verbatim}
gdf['name'] = gdf['name'].str.title()
\end{verbatim}

\subsubsection{Data subsetting with loc and iloc}
Recall that \verb_loc_ is a label based locator and \verb_iloc_is an integer-based locator.

\textbf{Range Selection}: \verb_gdf.loc[100:105]_ will include every value it is passed \textit{100-105}. \verb_gdf.iloc[100:105]_ will give half-open range (omitting the final value), \textit{100-104}

\textbf{loc with boolean selection}: 
We can return all words that start with 'E' using the \verb_str.startswith('E')_ attribute.

\subsection{Combining with Numpy Methods}
As an example we can use the Numpy "logical and" attribute for elementwise boolean selection.
 

\begin{verbatim}
np.logical_and
\end{verbatim}


Example syntax:

\begin{verbatim}
ed_names = gdf.loc[np.logical_and(gdf['name'].str.startswith
('E'), gdf['name'].str.endswith('d'))]
\end{verbatim}

\end{document}

